\documentclass[conference]{IEEEtran}
\IEEEoverridecommandlockouts
% The preceding line is only needed to identify funding in the first footnote. If that is unneeded, please comment it out.
\usepackage{cite}
\usepackage{amsmath,amssymb,amsfonts}
\usepackage{algorithmic}
\usepackage{graphicx}
\usepackage{textcomp}
\usepackage{xcolor}
\def\BibTeX{{\rm B\kern-.05em{\sc i\kern-.025em b}\kern-.08em
    T\kern-.1667em\lower.7ex\hbox{E}\kern-.125emX}}
\begin{document}

\title{Statistical Machine Learning Final Project\\
    {\large solved with RandomForest and ...}
}

\author{\IEEEauthorblockN{1\textsuperscript{st} Tilman Marquart}
\IEEEauthorblockA{\textit{Computer Science Bachelor} \\
\textit{SDU Exchange Student}\\
Nuremberg, Germany \\
timar20@student.sdu.dk}
\and
\IEEEauthorblockN{2\textsuperscript{nd} Given Name Surname}
\IEEEauthorblockA{\textit{dept. name of organization (of Aff.)} \\
\textit{name of organization (of Aff.)}\\
City, Country \\
email address or ORCID}
\and
\IEEEauthorblockN{3\textsuperscript{rd} Given Name Surname}
\IEEEauthorblockA{\textit{dept. name of organization (of Aff.)} \\
\textit{name of organization (of Aff.)}\\
City, Country \\
email address or ORCID}
}

\maketitle

\begin{abstract}
final project abstract, use random forest and ...
\end{abstract}

\section{Introduction}
Give a brief intuitive summary of the two chosen classification algorithms. Identify the critical parameters.

Talk about random forest with PCA. => critical hyperparams: PCA, NTREE, MTRY, NODESIZE, SAMPSIZE

\subsection{Dataset}
why corner and not mid? => on a visual inspection corner seemed to be better centered and the digits better captured

\section{Preprocessing}
Describe your preprocessing (dpi, PCA, centering, smoothing, normalization). Also show images.

\subsubsection{Random Forest Preprocessing}
why image wise min max norm? => some images with low saturation, digits probably written with a bright pencil
why 100 dpi? => one goal was to be fast, small image size will increase fastness

\subsection{Hyper parameter Optimization}
Optimize critical parameters (discussed under 1) on a smaller set and document this optimization process.

\subsubsection{Random Forest}
problem: on a smaller set!!! => maybe say smaller set was 30people and use crossvalidation with 43 to 4 with bigger set => argue why 30people was not a problem, because of colab and kaggle we choose 30people and let it run over night

explain grid search for: PCA, NTREE, MTRY, NODESIZE, SAMPSIZE
=> manual hyperparam optimization (by interpreting grid search barplots)
=> preprocess optimization (by using image wise min-max norm)
(remeber: Give information about the computational time required.)


\section{Evaluation}
Do a proper cross validation and indicate also mean and variances for all problems. Describe results on test and trainings set and reflect on overfitting. 
(remeber: Give information about the computational time required.)
(remeber: Analyze the results and give proper explanations)




\subsection{Random Forest}
=> use 10 fold crossvalidation with 43 to 4 with bigger set

(remeber: Give information about the computational time required.)
(remeber: Analyze the results and give proper explanations)
\section{Future Work}
Give an indication what could be further improved. 


\begin{thebibliography}{00}
\end{thebibliography}

\end{document}
